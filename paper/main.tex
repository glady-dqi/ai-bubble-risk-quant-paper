\documentclass[11pt]{article}
\usepackage[margin=1in]{geometry}
\usepackage{graphicx}
\usepackage{booktabs}
\usepackage{hyperref}
\title{Probability of an AI Bubble Burst: Is There a Bubble Today, How Will It End, and When?}
\author{Gladys (Research)}
\date{February 2026}
\begin{document}
\maketitle

\begin{abstract}
We quantify whether AI-related equities exhibit bubble-like dynamics and estimate the probability of a severe drawdown at 3, 6, and 12 months. We implement three complementary layers: (i) explosive root tests (SADF/GSADF), (ii) a nonlinear LPPL model to estimate critical-time dynamics, and (iii) a probabilistic drawdown model with walk-forward validation. Using daily prices (2015--2026) for a replicable AI beneficiary basket and non-AI benchmarks, we find statistically suggestive evidence of explosive episodes and a low but non-zero near-term drawdown risk. We separate statistical evidence from economic interpretation and document limitations.
\end{abstract}

\section{Introduction}
The AI narrative has driven sharp repricing in large-cap technology and semiconductor beneficiaries. This paper assesses whether current dynamics are consistent with a bubble and estimates the timing and probability of a burst. We define a ``burst'' as a $\ge$20\% drawdown over the next 3 months, and we triangulate evidence using explosive-root tests, LPPL critical-time estimation, and a calibrated probabilistic model.

\section{Related Literature}
Our methodology follows Phillips, Shi, and Yu (2015) for bubble detection via explosive-root tests, and the LPPL literature for nonlinear bubble dynamics. We complement with probabilistic classification of drawdown risk using walk-forward validation, avoiding look-ahead bias.

\section{Data and Proxies}
\textbf{AI Universe (replicable):} NVDA, MSFT, GOOGL, AMZN, META, AAPL, TSLA, AMD, AVGO, ASML, SMH.\
\textbf{Benchmarks:} SPY, QQQ, XLK, SOXX.\
Daily adjusted prices (2015--2026) are sourced from Yahoo Finance. We construct an equal-weight AI basket and use SPY as a non-AI benchmark. Proxies include 1--3 month momentum, 3-month realized volatility, and 12-month relative performance vs. SPY.

\section{Methodology}
\subsection{Explosive-Root Tests (SADF/GSADF)}
We compute SADF and GSADF on log-price of the AI basket. Positive test statistics imply evidence against a unit root in favor of explosiveness. We report statistics and interpret them relative to the right-tail of the ADF distribution, noting that exact critical values require bootstrapping.

\subsection{LPPL}
We fit a LPPL model on the last 500 trading days to estimate critical-time dynamics and super-exponential growth. The estimated critical time is reported in relative units (days ahead in the window).

\subsection{Probabilistic Drawdown Model}
We build a logistic model using momentum, volatility, and relative performance features. Labels are defined as a future drawdown $\ge$20\% over horizons 3, 6, and 12 months. We use walk-forward validation with a 2022+ test period to avoid look-ahead.

\section{Results}
\begin{itemize}
\item \textbf{Explosive tests:} SADF = 0.71; GSADF (sub-sampled) = 1.00. Both are positive, suggesting episodic explosiveness. Critical values are not bootstrapped here, so we treat these as indicative rather than definitive.
\item \textbf{LPPL:} The fitted critical-time parameter implies elevated risk in the medium term (1--2 years), conditional on continuation of the current regime.
\item \textbf{Crash probabilities:} The 3-month logistic model yields a mean probability of 0.79\% and a latest estimate of 1.19\%. For 6- and 12-month horizons we report empirical (unconditional) frequencies of 20\% drawdowns: 6m = 3.08\%, 12m = 3.51\%. These longer-horizon frequencies should be interpreted as baseline risk, not conditional forecasts.
\end{itemize}

\begin{figure}[h]
\centering
\includegraphics[width=0.85\linewidth]{../figures/rolling_adf.png}
\caption{Rolling ADF (200-day) for the AI basket. Crossings above the threshold indicate local explosiveness.}
\end{figure}

\begin{figure}[h]
\centering
\includegraphics[width=0.85\linewidth]{../figures/crash_prob_3m.png}
\caption{Predicted 3-month probability of a 20\% drawdown (walk-forward logistic model).}
\end{figure}

\section{Economic Interpretation vs. Statistical Evidence}
\textbf{Statistical evidence:} Positive SADF/GSADF and LPPL curvature indicate episodes consistent with bubble-like dynamics.\
\textbf{Economic interpretation:} These signals coincide with rapid repricing and elevated concentration in AI beneficiaries, but do not imply an imminent crash.\
\textbf{Limitations:} (i) GSADF is sub-sampled for computational feasibility; (ii) critical values are not bootstrapped; (iii) 3m crash probabilities are model-based and should be interpreted as relative risk indicators, while 6/12m are unconditional baselines.

\section{Robustness}
We performed sub-sample checks and alternative rolling windows (200--400 days). Results are robust in sign but sensitive to volatility and relative performance inputs. Walk-forward validation mitigates look-ahead bias.

\section{Conclusion}
The evidence suggests episodic exuberance in AI beneficiaries, with a low but non-zero probability of a severe drawdown in the next 3--12 months. The most plausible outcome is a soft landing; a crash scenario is more likely under macro shocks (rates or earnings). Monitoring volatility and relative performance remains crucial.

\section*{References}
Phillips, P.C.B., Shi, S., Yu, J. (2015). Testing for multiple bubbles.

\section*{Appendix}
Event definition: drawdown $\ge$20\% over 3/6/12 months.\\
Models estimated with walk-forward validation (train $\le$ 2021-12, test $\ge$ 2022-01).

\end{document}
