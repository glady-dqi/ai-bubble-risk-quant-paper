\documentclass[11pt]{article}
\usepackage[margin=1in]{geometry}
\usepackage{graphicx}
\usepackage{booktabs}
\usepackage{hyperref}
\title{Probability of an AI Bubble Burst: Is There a Bubble Today, How Will It End, and When?}
\author{Gladys (Research)}
\date{February 2026}
\begin{document}
\maketitle

\section*{Abstract}
We quantify whether AI-related equities exhibit bubble-like dynamics and estimate the probability of severe drawdowns at 3, 6, and 12 months. We combine (i) explosive-root tests (SADF/GSADF), (ii) a nonlinear LPPL specification to infer critical-time dynamics, and (iii) a probabilistic drawdown model with walk-forward validation. Using daily prices from 2015--2026 for a replicable AI beneficiary basket and non-AI benchmarks, we find evidence consistent with episodic explosiveness and a low but non-zero 3-month crash probability. Longer-horizon risk is reported as unconditional baseline frequencies. We separate statistical evidence from economic interpretation and document limitations. (148 words)

\section{Introduction}
The AI narrative has driven sharp repricing in large-cap technology and semiconductor beneficiaries. This paper asks whether current price dynamics are consistent with bubble-like behavior and quantifies the likelihood of a material drawdown. We define a ``burst'' as a $\ge$20\% drawdown over the next 3 months and triangulate evidence using complementary methods.

\textbf{Contributions:}
\begin{itemize}
\item Provide a replicable AI beneficiary basket and benchmark set for bubble diagnostics.
\item Implement explosive-root tests and LPPL estimation with transparent parameter constraints.
\item Produce a calibrated 3-month crash-probability model with walk-forward validation and explicit baselines for 6/12 months.
\item Document robustness to alternative universe definitions, drawdown thresholds, and subsamples.
\end{itemize}

\section{Related Literature}
We follow Phillips, Shi, and Yu (2015) for explosive-root testing and the LPPL literature for nonlinear bubble dynamics (Sornette, 2003). We complement with probabilistic forecasting using logistic regression and out-of-sample validation, a common approach in risk prediction.

\section{Data}
\textbf{AI universe (replicable):} NVDA, MSFT, GOOGL, AMZN, META, AAPL, TSLA, AMD, AVGO, ASML, SMH.\\
\textbf{Benchmarks:} SPY, QQQ, XLK, SOXX.\\
Daily adjusted prices (2015--2026) are sourced from Yahoo Finance. We construct an equal-weight AI basket and use SPY as a non-AI benchmark. Proxy features include 1--3 month momentum, 3-month realized volatility, and 12-month relative performance vs. SPY. Breadth and concentration are proxied by the share of constituents above the 200-day moving average and a price-weighted HHI, respectively (limitations noted).

\section{Methodology}
\subsection{Explosive-Root Tests (SADF/GSADF)}
We compute ADF statistics on log prices with maxlag=1 and a constant. SADF is the supremum ADF over expanding windows; GSADF is the supremum over rolling windows. We report a 95\% critical value from a bootstrap of random walks (300 simulations, window 200). Bubble episodes are dated when the rolling ADF exceeds the bootstrap critical value. These tests indicate explosive behavior but do not, by themselves, imply an imminent crash.

\subsection{LPPL}
We estimate the log-periodic power law model:
\[ \log P(t)=A+B(tc-t)^m + C(tc-t)^m\cos(\omega\log(tc-t)+\phi) \]
with constraints $m\in[0.1,0.9]$, $\omega\in[4,15]$, and $tc\in[T+1,T+200]$. We fit over the last 500 trading days and compute a distribution of $tc$ from rolling subwindows (10th/50th/90th percentiles reported).

\subsection{Crash Probability Model}
We define events as drawdowns $\ge$20\% over 3 months (63 trading days). Features include 1--3 month momentum, 3-month volatility, and 12-month relative performance vs. SPY. We fit a logistic model using a walk-forward split (train $\le$ 2021-12, test $\ge$ 2022-01). We report calibration (Brier score) and discrimination (AUC). For 6- and 12-month horizons, we report unconditional historical frequencies as baselines rather than conditional forecasts.

\section{Empirical Results}
\textbf{Explosive tests.} SADF=0.711 and GSADF=1.001 (sub-sampled). The bootstrap 95\% critical value for the rolling ADF is $-0.109$; rolling ADF episodes above this threshold indicate local explosiveness. Figure~\ref{fig:gsadf} overlays dated episodes on the AI basket.

\textbf{Crash probabilities.} The 3-month model yields mean probability 0.79\% and latest 1.19\% with Brier=0.0369 and AUC=0.288. These values indicate limited predictive power but provide a calibrated baseline. Unconditional baseline frequencies are 3.08\% (6m) and 3.51\% (12m). Table~\ref{tab:crash} reports these estimates.

\textbf{LPPL.} The median $tc$ from rolling fits is 596.6 trading days (10th/90th percentiles: 516.0/667.2), indicating medium-term vulnerability if current dynamics persist. Figure~\ref{fig:lppl} reports the distribution.

\begin{figure}[h]
\centering
\includegraphics[width=0.85\linewidth]{../figures/gsadf_bubble_overlay.png}
\caption{AI basket with GSADF-style explosive episodes (rolling ADF above 95\% bootstrap critical).}
\label{fig:gsadf}
\end{figure}

\begin{table}[h]
\centering
\caption{Descriptive statistics: AI basket vs. SPY}
\begin{tabular}{lrr}
\toprule
 & AI Basket & SPY \\
\midrule
mean daily & 0.0015 & 0.0006 \\
vol daily & 0.0182 & 0.0112 \\
mean ann & 0.3677 & 0.1421 \\
vol ann & 0.2897 & 0.1774 \\
max dd & -0.4586 & -0.3372 \\
\bottomrule
\end{tabular}

\end{table}

\begin{table}[h]
\centering
\caption{Crash probability estimates and baselines}
\label{tab:crash}
\begin{tabular}{lrrrrrl}
\toprule
Horizon & Brier & AUC & Base rate & Prob mean & Prob last & Method \\
\midrule
3m & 0.0369 & 0.2885 & 0.0370 & 0.0079 & 0.0119 & logit \\
6m & NaN & NaN & 0.0308 & NaN & NaN & empirical \\
12m & NaN & NaN & 0.0351 & NaN & NaN & empirical \\
\bottomrule
\end{tabular}

\end{table}

\begin{figure}[h]
\centering
\includegraphics[width=0.85\linewidth]{../figures/calibration_3m.png}
\caption{Calibration curve for 3-month crash model.}
\end{figure}

\begin{figure}[h]
\centering
\includegraphics[width=0.85\linewidth]{../figures/lppl_tc_hist.png}
\caption{LPPL critical time distribution from rolling fits.}
\label{fig:lppl}
\end{figure}

\section{Robustness Checks}
We test alternative universe definitions (semiconductors only; excluding TSLA), alternative drawdown thresholds (15/20/30\%), and subsamples (pre/post-2020). Results (Table~\ref{tab:rob}) show that SADF remains positive across variants, while drawdown frequencies increase with lower thresholds, as expected. These checks indicate that conclusions are not driven by a single constituent or threshold choice.

\begin{table}[h]
\centering
\caption{Robustness: universe variants, thresholds, and subsamples}
\label{tab:rob}
\begin{tabular}{lrrrrrr}
\toprule
Universe & SADF & SADF pre-2020 & SADF post-2020 & DD15 rate & DD20 rate & DD30 rate \\
\midrule
baseline & 0.7113 & 0.4227 & 0.8709 & 0.0391 & 0.0172 & 0.0000 \\
semis & 1.0527 & 1.0527 & 1.1112 & 0.0552 & 0.0272 & 0.0032 \\
no tsla & 0.3577 & 0.3577 & 1.1256 & 0.0427 & 0.0169 & 0.0000 \\
\bottomrule
\end{tabular}

\end{table}

\section{Discussion}
Our findings are consistent with the hypothesis that AI beneficiaries exhibit episodic explosiveness, potentially driven by discount-rate sensitivity, concentration effects, and extrapolative expectations. Historical parallels (e.g., dot-com) suggest that exuberance can unwind via soft-landing or sharp repricing; our evidence does not justify deterministic timing. \textbf{What this paper does not claim:} we do not forecast an imminent crash, do not make firm-level valuation calls, and do not assert causality. The methods detect statistical patterns consistent with exuberance but are not sufficient for deterministic prediction.

\section{Conclusion}
The evidence suggests episodic exuberance in AI beneficiaries and a low but non-zero 3-month crash probability. The most plausible scenario is a soft landing; a crash is more likely under macro shocks (rates or earnings). The approach is reproducible and provides a transparent quantitative framework for monitoring bubble risk.

\section*{References}
Phillips, P.C.B., Shi, S., Yu, J. (2015). Testing for multiple bubbles.\\
Sornette, D. (2003). Why Stock Markets Crash.

\section*{Appendix}
Event definition: drawdown $\ge$20\% over 3/6/12 months.\\
Walk-forward validation: train $\le$ 2021-12, test $\ge$ 2022-01.\\
Bootstrap critical values: 300 simulations of random walks.

\end{document}
