\documentclass[11pt]{article}
\usepackage[margin=1in]{geometry}
\usepackage{graphicx}
\usepackage{booktabs}
\usepackage{hyperref}
\title{AI-Specific Exuberance or Factor Exposure? Evidence from Bubble Diagnostics and Residualized Prices}
\author{Gladys (Research)}
\date{February 2026}
\begin{document}
\maketitle

\section{Abstract}
We test whether apparent ``AI bubble'' dynamics reflect AI-specific exuberance or exposure to common tech and market factors. We implement (i) explosive-root tests (SADF/GSADF), (ii) a nonlinear LPPL model, and (iii) a calibrated 3-month crash-risk model. Crucially, we re-run diagnostics on factor-residualized AI prices and on matched non-AI control universes. Using daily data (2015--2026), we find strong explosiveness in raw AI prices but weaker, though still positive, explosiveness in residualized series. This pattern suggests that a portion of the signal is AI-specific, while common factor exposure explains a meaningful share. We report crash probabilities as calibrated monitoring signals and provide robustness across universes and thresholds. (149 words)

\section{Introduction}
The AI narrative has driven rapid repricing in large-cap technology and semiconductor beneficiaries. The central identification question is whether this behavior reflects AI-specific exuberance or simply exposure to common tech/market and duration factors. A ``burst'' is defined as a $\ge$20\% drawdown over the next 3 months.

\textbf{Why now.} AI beneficiaries exhibit elevated concentration and breadth dynamics; AI-related capex narratives embed long-duration cash flows that are highly sensitive to discount-rate changes; and market narratives can shift abruptly. Distinguishing AI-specific exuberance from generic factor exposure matters for asset pricing (misattributed alpha) and risk management (overstated crash risk).

\textbf{Literature gap.} Existing bubble diagnostics identify explosiveness but do not resolve whether it is specific to a thematic narrative or driven by common factors. This paper bridges that gap by applying the same diagnostics to factor-residualized prices and matched non-AI controls, enabling an identification-style comparison.

\textbf{Contributions.}
\begin{itemize}
\item We introduce a replicable identification design: AI vs non-AI matched controls, and raw vs factor-residualized price processes.
\item We implement explosive-root and LPPL diagnostics with explicit parameter constraints and bootstrap critical values, enabling dated episodes rather than narrative inference.
\item We provide a calibrated crash-risk layer and interpret it conditionally on whether exuberance persists after factor adjustment.
\item We document robustness to alternative universe definitions, drawdown thresholds, and subsamples, clarifying stable vs sensitive findings.
\end{itemize}

\section{Related Literature}
\subsection{Bubble detection: SADF/GSADF}
Explosive-root tests (Phillips, Shi, and Yu, 2015) detect transient explosive behavior but do not imply imminent crashes or fundamental mispricing. We extend their use by comparing raw and residualized AI series and matched controls.

\subsection{Nonlinear bubble models: LPPL/LPPLS}
LPPL models (Sornette, 2003) estimate super-exponential dynamics and critical-time parameters; these are sensitive to window choice and constraints. We adopt conservative bounds and report a distribution of critical times from rolling fits to express uncertainty.

\subsection{Crash risk and predictability}
Rare-event forecasting often emphasizes calibration rather than discrimination. We report Brier and AUC metrics and compare against naive baselines, interpreting probabilities as monitoring signals rather than point forecasts.

\section{Data}
\textbf{AI universe:} NVDA, MSFT, GOOGL, AMZN, META, AAPL, TSLA, AMD, AVGO, ASML, SMH.\\
\textbf{Non-AI tech controls:} IBM, ORCL, CSCO, INTC, TXN, QCOM, ADBE.\\
\textbf{AI semiconductors:} NVDA, AMD, AVGO, ASML, SMH.\\
\textbf{Non-AI semiconductors:} INTC, TXN, QCOM, MU, NXPI.\\
\textbf{Benchmarks:} SPY, QQQ, XLK.\\
Daily adjusted prices (2015--2026) are sourced from Yahoo Finance. We construct equal-weight baskets for each universe. Factors include SPY and XLK returns, and changes in 10Y yield (\^{TNX}) and VIX (\^{VIX}).

\section{Methodology}
\subsection{Explosive-Root Tests (SADF/GSADF)}
We compute ADF statistics on log prices with maxlag=1 and a constant. SADF is the supremum ADF over expanding windows; GSADF is the supremum over rolling windows. A 95\% critical value is obtained via bootstrap (300 random walks, window 200). Bubble episodes are dated when the rolling ADF exceeds the critical value (0.0019 in our sample). The bootstrap critical value may be positive or negative depending on sample; ADF statistics above this right-tail threshold are consistent with explosiveness but do not imply a deterministic crash.

\subsection{Factor Residualization (Identification Layer)}
We estimate:
\[ r^{AI}_t = \alpha + \beta_M r^{SPY}_t + \beta_{Tech} r^{XLK}_t + \beta_{Rates}\Delta y_t + \beta_{Vol}\Delta VIX_t + \varepsilon_t. \]
Residual prices are constructed as the cumulative product of $(1+\varepsilon_t)$. We re-run SADF/GSADF and LPPL on residualized AI baskets and on residualized control universes. Persistence of explosiveness in residuals indicates AI-specific dynamics; attenuation indicates factor-driven behavior.

\subsection{LPPL}
We estimate
\[ \log P(t)=A+B(tc-t)^m + C(tc-t)^m\cos(\omega\log(tc-t)+\phi), \]
with $m\in[0.1,0.9]$, $\omega\in[4,15]$, and $tc\in[T+1,T+200]$. We fit over the last 500 trading days and report 10th/50th/90th percentiles of $tc$ from rolling subwindows, interpreting these as conditional vulnerability windows.

\subsection{Crash Probability Model}
Events are drawdowns $\ge$20\% over 3 months (63 trading days). Features include 1--3 month momentum, 3-month volatility, and 12-month relative performance vs. SPY. We fit a logistic model using walk-forward validation (train $\le$ 2021-12, test $\ge$ 2022-01) and report Brier and AUC. For 6/12 months we report unconditional baseline frequencies.

\section{Empirical Results}
\subsection{Raw diagnostics: AI vs benchmarks}
Raw AI diagnostics show SADF=0.711 and GSADF=1.001 (sub-sampled). Figure~\ref{fig:gsadf} overlays explosive episodes on the AI basket, and Figure~\ref{fig:controls} benchmarks AI vs matched controls. Relative to broad benchmarks, explosiveness is more pronounced in the AI basket.

\subsection{Residualized diagnostics: identification}
After factor residualization, AI explosiveness attenuates but remains positive (SADF=-0.443, GSADF=0.618). Non-AI tech residuals are weaker (GSADF=0.190), and non-AI semis show smaller residual explosiveness. Table~\ref{tab:gsadf} compares raw vs residualized series. This pattern suggests that common factors explain part—but not all—of the AI exuberance signal.

\subsection{Concentration and breadth}
AI baskets exhibit higher breadth and distinct concentration dynamics relative to non-AI controls. Table~\ref{tab:conc} reports mean and last values for breadth, dispersion, and return concentration. These cross-sectional patterns are consistent with narrative-driven concentration but do not establish causality.

\subsection{Implications for crash-risk probabilities}
The 3-month crash model yields mean probability 0.79\% and latest 1.19\% with Brier=0.0369 and AUC=0.288. The low AUC indicates limited discrimination; we interpret probabilities as calibrated monitoring signals relative to the baseline rate (3.70\%). Unconditional baselines are 3.08\% (6m) and 3.51\% (12m). Table~\ref{tab:crash} and Figure~\ref{fig:cal} summarize calibration.

\begin{figure}[h]
\centering
\includegraphics[width=0.85\linewidth]{../figures/gsadf_bubble_overlay.png}
\caption{AI basket with explosive episodes (rolling ADF above 95\% bootstrap critical).}
\label{fig:gsadf}
\end{figure}

\begin{figure}[h]
\centering
\includegraphics[width=0.85\linewidth]{../figures/ai_vs_controls.png}
\caption{AI vs matched controls (indexed price paths).}
\label{fig:controls}
\end{figure}

\begin{table}[h]
\centering
\caption{GSADF comparison: raw vs residualized series}
\label{tab:gsadf}
\begin{tabular}{lrr}
\toprule
Series & SADF & GSADF \\
\midrule
AI raw & 0.7113 & 1.0006 \\
AI residual & -0.4429 & 0.6182 \\
Non-AI tech raw & 0.6804 & -0.2590 \\
Non-AI tech residual & 0.1463 & 0.1899 \\
AI semis raw & 1.0527 & 1.2113 \\
AI semis residual & 0.6421 & 0.9612 \\
Non-AI semis raw & 0.6787 & 1.5459 \\
Non-AI semis residual & 1.5316 & 1.5673 \\
\bottomrule
\end{tabular}

\end{table}

\begin{table}[h]
\centering
\caption{Concentration and breadth metrics}
\label{tab:conc}
\begin{tabular}{lrrrrrr}
\toprule
Universe & Breadth mean & Breadth last & Dispersion mean & Dispersion last & HHI mean & HHI last \\
\midrule
AI basket & 0.7131 & 0.7273 & 0.0160 & 0.0210 & 0.1570 & 0.1761 \\
Non-AI tech & 0.6127 & 0.5714 & 0.0122 & 0.0392 & 0.2266 & 0.2799 \\
AI semis & 0.7338 & 1.0000 & 0.0141 & 0.0194 & 0.2963 & 0.3612 \\
Non-AI semis & 0.5758 & 0.8000 & 0.0136 & 0.0381 & 0.3029 & 0.4821 \\
\bottomrule
\end{tabular}

\end{table}

\begin{table}[h]
\centering
\caption{Crash probability estimates and baselines}
\label{tab:crash}
\begin{tabular}{lrrrrrl}
\toprule
Horizon & Brier & AUC & Base rate & Prob mean & Prob last & Method \\
\midrule
3m & 0.0369 & 0.2885 & 0.0370 & 0.0079 & 0.0119 & logit \\
6m & NaN & NaN & 0.0308 & NaN & NaN & empirical \\
12m & NaN & NaN & 0.0351 & NaN & NaN & empirical \\
\bottomrule
\end{tabular}

\end{table}

\begin{figure}[h]
\centering
\includegraphics[width=0.85\linewidth]{../figures/calibration_3m.png}
\caption{Calibration curve for 3-month crash model.}
\label{fig:cal}
\end{figure}

\begin{figure}[h]
\centering
\includegraphics[width=0.85\linewidth]{../figures/lppl_tc_hist.png}
\caption{LPPL critical time distribution from rolling fits.}
\label{fig:lppl}
\end{figure}

\section{Robustness Checks}
We test alternative universe definitions (semiconductors only; excluding TSLA), alternative drawdown thresholds (15/20/30\%), and subsamples (pre/post-2020). SADF remains positive across variants, and drawdown frequencies shift monotonically with thresholds. These results indicate stability in explosiveness signals but sensitivity in drawdown frequencies. Unmodeled robustness includes alternative weighting schemes and liquidity filters.

\section{Discussion}
The evidence supports partial AI-specific exuberance: explosiveness weakens after factor residualization but does not disappear. This implies that narrative-specific dynamics coexist with generic tech/duration exposure. Investors who attribute all AI price behavior to a thematic bubble risk mislabeling factor beta as ``AI.'' \textbf{What this paper does not claim:} it does not forecast an imminent crash, does not issue firm-level valuation calls, and does not assert causal mechanisms.

\section{Conclusion}
We identify AI-specific exuberance as a residual component after accounting for broad market, tech, rate, and volatility factors. The framework pairs bubble diagnostics with calibrated risk monitoring and explicit controls. Practically, it supports scenario-aware risk management rather than point prediction.

\section*{References}
Phillips, P.C.B., Shi, S., Yu, J. (2015). Testing for multiple bubbles.\\
Sornette, D. (2003). Why Stock Markets Crash.\\
Campbell, J.Y., and Cochrane, J.H. (1999). By force of habit.\\
Kelly, B., and Jiang, H. (2014). Tail risk and return predictability.\\
Merton, R.C. (1976). Option pricing when underlying returns are discontinuous.\\

\section*{Appendix}
Event definition: drawdown $\ge$20\% over 3/6/12 months.\\
Walk-forward validation: train $\le$ 2021-12, test $\ge$ 2022-01.\\
Bootstrap critical values: 300 simulations of random walks.\\
Identification: residualized series constructed from factor regression residuals.

\end{document}
